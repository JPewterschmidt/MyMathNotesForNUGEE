\chapter{线性代数第二部分简明笔记} 

The second part is about \textbf{change}.
Time enters the picture --- continuous time in a differential
equation $\mathbf{\mathrm du}/\mathrm dt = A \mathbf u$ or
time steps in a difference equation $\mathbf u_{k + 1} = A\mathbf u_k$.
Those functions are \textbf{NOT} solved by elimination.
\cite[Chapter 6]{strang}.

\section{Eigen-stuff}

特征向量 $x$ 是一个方向和 $Ax$ 相同的向量
\footnote{方向相同,但其模大小可能会变化(有伸缩)}
,伸缩量就是 $\lambda$,也就是特征值。
也就是说 
\begin{gather*}
    Ax = \lambda x \\
    (A - \lambda I) x = 0
\end{gather*}
零向量不能是特征向量。也就是说有一个对于矩阵 $(A - \lambda I)$ 来说,
存在非零线性组合结果为0,进而 $(A - \lambda I)$ 是奇异的。
所以
\[
    \det [A - \lambda I] = 0
\]
可以通过上等式求出多个 $\lambda$.

\textbf{Summary:}
\begin{enumerate}
    \item Compute the determinant of $A - \lambda I$
    \item Find the roots of this polynomial, 
    \item solve $(A - \lambda I) x = 0$ to find an eigenvector $x$.
\end{enumerate}

虽然特征向量不能为0,但是特征值可以为0,此时特征向量在 Null space 中。

针对一些特殊矩阵,他们的特征值往往是有迹可循的。
记忆这些内容对应试有一些好处,但是要结合这些特殊矩阵的性质来记忆。
\begin{itemize}
    \item When $A$ is \textbf{squared} the eigenvectors stay the same, the \textbf{eigenvalues are squared}.
        This pattern keeps going, because the eigenvectors stay in their own directions and never get mixed.
        Other vectors do change direction. But all others are combinations of the two eigenvectors.
    \item \textbf{Singular matrix} has a eigenvalue $0$, 
        but you still has to find the other one and eigenvectors 
        by the determinant. And if $A$ is invertible, zero is not an eigenvalue.
        We shift $A$ by a multiple of $I$ to \emph{make it singular}.
    \item \textbf{Markov Matrix}: Each column of $P$ adds to 1, so $\lambda = 1$ is an eigenvalue.
    \item \textbf{$P$ is singular}, so $\lambda = 0$ is an eignevalue. 
    \item \textbf{$P$ is symmetric\footnote{TODO}}, so its eigenvectors $(1, 1)$ and $(1, -1)$ are prependicular.
    \item The only eignevalues of \textbf{projection matrix} are $0$ and $1$. 
        \begin{itemize}
            \item The eigenvalue $\lambda = 0$ (which means $Px = 0x$) fill up the null space;
            \item The eignevalue $\lambda = 1$ (which means $Px = x$)  fill up the column space.
        \end{itemize}
        The zero space is projected onto zero, the column space is projected onto itself.
        \uline{The projection keeps the column space and destroys the nullspace.}
    \item \uline{Elimination does not preserve the $\lambda$'s, }
        The \textbf{triangular\footnote{包括上下三角矩阵} $U$ and diagonal $D$} 
        have their eigenvalues sitting along the diagonal 
        -- they are the pivots. But they're not the eigenvalues of $A$.
    \item The product of the $n$ eignevalues \emph{equals} to the determinant. 
    \item The sum of the $n$ eigenvalues \emph{equals} the sum of the $n$ diagonal entries.
\end{itemize}
后两条是比较有用的,但是这并不能代替你求 $\lambda$'s 的工作,
可以用这样的方法来检验你的正确性。
\[
    \lambda_1 + 
    \lambda_2 + 
    \lambda_3 + 
    \cdots + 
    \lambda_n = 
    \mbox{\textbf{trace}} = 
    a_{11} + 
    a_{22} + 
    a_{33} + 
    \cdots
    a_{nn} 
\]

例题参见 \cite[page 295]{strang}.

\section{Diagonalizing a Matrix}

对角化的好处是十分明显的,可以很大程度上简化计算。
\begin{gather*}
    X^{-1} AX = \Lambda \\
    A = X\Lambda X^{-1}
\end{gather*}

如此有
\[
    A^2 =X\Lambda X^{-1}X\Lambda X^{-1} = X\Lambda^2X^{-1}
\]
The matrix $X$ has an inverse, because its columns (the eigenvectors of $A$)
were assumed to be linearity independent. 
\emph{Without $n$ independent eigenvectors, we can't diagonalize.}

\textbf{Remarks of $\Lambda$:}
\begin{itemize}
    \item Suppose the eigenvalues $\lambda_1, \lambda_2, \lambda_3, \cdots, \lambda_n$ are all different.
        Then it is automatic that the eigenvectors $x_1, x_2, x_3, \cdots, x_n$ are independent.
        This eigenvector matrix $X$ will be \emph{invertible},
        \textbf{\uline{Any matrix that has no repeated eigenvalues can be diagonalized!}}
    \item \textbf{
        We can multiply eigenvectors by any nonzero constants
        \footnote{
            考研数学在卷面书写的时候似乎要体现这一点,比如将答案写成 
            $\cramped{k(a, b, c)^T, \quad k \neq 0.}$
        }.
    }
    \item The eigenvectors in $X$ come in the same order as the eigenvalues in $\Lambda$.
    \item (repeated wanring for repeated eigenvectors) 
        Some matrices have too few eigenvectors. 
        \emph{Those matrices connot be diagonalized}\cite[page 305]{strang}.
        But remember that there is no connection 
        between \emph{invertibility} and \emph{diagonalizability}
        \footnote{
            For example, the identity has all $\lambda = 1$, but it's Diagonalizabe.
        }:\begin{itemize}
            \item \textbf{Invertibility} is concernd 
                with the eigenvalues ($\lambda = 0$ or $\lambda \neq 0$).
            \item \textbf{Diagonalizability} is concernd 
                with the eigenvectors (too few or enough for $X$).
        \end{itemize}
\end{itemize}

\section{Symmetric Matrices}

\begin{itemize}
    \item A symmetric matrix $S$ has $n$ \textbf{real eigenvalues} $\lambda_i$ 
        and $n$ \textbf{orthonormal eigenvectors} $q_1, \cdots, q_n$.
    \item Every real symmetric
        \footnote{$Q^{-1} = Q^T$ 是正交阵的特性}
        $S$ can be diagonalized:
        \[
            S = Q\Lambda Q^{-1} = Q\Lambda Q^T
        \]
    \item The number of positive eigenvalues of $S$ equals the number of positive \textbf{pivots}.
    \item Antisymmetric matrices 
        $A = -A^T$ have \emph{imaginary} $\lambda$'s 
        and \emph{orthonormal (complex)} $q$'s.
\end{itemize}
