\PassOptionsToPackage{quiet}{xeCJK}
\documentclass{beaulivre}

%\usepackage{multicol}
\usepackage{booktabs}
\usepackage{cancel}
\usepackage{caption}
\usepackage{makeidx}
\usepackage{geometry}
\usepackage{sidenotes}
\usepackage{amsmath}
\usepackage{amssymb}
\usepackage{hyperref}
\usepackage{cuted}
\usepackage{amsthm}
\usepackage{ulem}
\geometry{a4paper}

\bibliographystyle{alpha}

\title{我的考研数学笔记}
\author{王世鑫\thanks{E-mail:w2683107566@live.com}}
\date{\today}

\begin{document}

%\maketitle

\frontmatter
在科学上面没有平坦的大道,只有不畏劳苦沿着陡峭的山路攀登的人,才有希望达以光辉的顶点。

这是我第一次正式使用 \LaTeX{} 来进行写作。作为我的考研数学复习的笔记之用。
作笔记本身是对知识的一种复盘和总结,使用 \LaTeX{} 可以对这个过程增添几分乐趣。
某种程度上能和我同期正在缓慢进行的个人C++项目 koios 一样,在枯燥乏味的考研备考生活中增添几分色彩。

这本笔记将会收录一些在各类成体系的考研教学视频以及和考研数学相关的文章、
视频中提到的做题方法和一些基本的定理以供本人成查阅。
由于内容是数学相关的内容,本人并非数学专业学生,因此在表述上难免出现令数学人嗤之以鼻的不严谨,
如果您发现了一些这样的现象,欢迎直接提PR对本笔记进行修改,
如果觉得麻烦,也烦请您直接以各种形式告知我,感激不尽。
我会在每周复盘进行笔记的时候对PR进行集中处理。

感谢我的考研宿舍舍友王超同学,在笔记撰写期间提供一些宝贵建议。

\tableofcontents

\mainmatter
\twocolumn
\chapter{全局通用技术}

\begin{lemma}
\begin{multline*}
	f, g \mbox{在} x_0 \mbox{有公共切线} \\ \Rightarrow f(x_0) = g(x_0), f'(x_0) = g'(x_0)
\end{multline*}
\end{lemma}
上述内容中小心遗漏 $f(x_0) = g(x_0)$

\begin{definition}
    \begin{equation}
        \mathbf{C}_{k}^{n}\binom{n}{k} = \frac{n!}{k!(n-k)!}
    \end{equation}
\end{definition}

\begin{lemma}
    \begin{align*}
        (a+b)^3 = a^3 + 3a^2b + 3ab^2 + b^3\\
        (a-b)^3 = a^3 - 3a^2b + 3ab^2 - b^3
    \end{align*}
\end{lemma}

\section{函数}

无论是导数还是积分,相关题目和函数性质(图像、单调性等)相关性总是很大,
在处理的时候要善于利用函数的性质,辅助解题。
利用函数性质辅助解题的有例子:
题目 
\begin{itemize}

    \item   \cite[page 70, pdf 81, 例1]{we} 和 
            \cite[page 75, pdf 86, 例6]{we}利用函数是偶函数,
            因此图像对称,从而你确定第二条渐近线。

    \item   题目 \cite[page 77, pdf 88, 例5, 例6]{we} 
            通过观察函数图像,辅助确定根的个数,
            有关根的个数请参阅 \ref{number-of-roots-question}。

    \item   题目 \cite[page 107, pdf 118, example 4]{we}
            认识到原函数是偶函数,进而简化求值过程。

\end{itemize}

\subsection{如何联系函数和导数}

TODO
%TODO

\section{Tylor Series and Formula} \label{tylor}

\subsection{公式和形式}

\begin{definition}[泰勒公式的一般形式]
	\begin{equation}
		f(x)=\sum_{i=0}^{n}{\frac{f^{(i)}(x_0)}{i!}(x-x_0)^i}+R_n(x)
	\end{equation}
	其中,$f^{(i)}(x_0)$表示$f(x)$在$x_0$处的$i$阶导数,$R_n(x)$表示余项,有多种形式,例如:
	\begin{align*}
		&R_n(x)=\frac{f^{(n+1)}(\xi)}{(n+1)!}(x-x_0)^{n+1}, \\ &\xi \in [x,x_0]\ or \ \xi \in [x_0,x]
	\end{align*}
\end{definition}

In summary, the Peano remainder focuses on 
the local behavior of the function near the point of expansion 
and gives a rough upper bound on the error for a specific value of $x$. 
\begin{definition}{The Cauchy remainder}
    \[
        R_n=\dfrac{(x-x^*)^n}{n!}(x-x_0)f^{(n+1)}(x^*)
    \]
    where $x^*$ is in $[x_0,x]$ (Hamilton 1952).
\end{definition}
The \textbf{Cauchy remainder provides a more global perspective by 
considering the behavior of derivatives across a broader 
interval between $x_0$ and $x$}. It gives a more refined estimate 
of the error over that interval. This distinction in the scope 
of accuracy makes the Cauchy remainder more useful when you want 
to understand how well the Taylor polynomial approximates the function 
across a larger range of values.

下面是常用的一些Tylor公式
\begin{align}
	\notag e^x&=1+x+\frac{x^2}{2!}+\frac{x^3}{3!} \\ &+\cdots+\frac{x^n}{n!}+o(x^n)\\
	\notag \ln(1+x)&=x-\frac{x^2}{2}+\frac{x^3}{3}\\ &-\cdots+(-1)^{n-1}\frac{x^n}{n}+o(x^n)\\
	\notag \sin x&=x-\frac{x^3}{3!}+\frac{x^5}{5!}\\ &-\cdots+(-1)^{n-1}\frac{x^{2n-1}}{(2n-1)!}+o(x^{2n})\\
	\notag \cos x&=1-\frac{x^2}{2!}+\frac{x^4}{4!}\\ &-\cdots+(-1)^n\frac{x^{2n}}{(2n)!}+o(x^{2n+1})\\
	\notag (1+x)^\alpha&=1+\alpha x+\frac{\alpha(\alpha-1)}{2!}x^2+ \\  &\cdots+\frac{\alpha(\alpha-1)\cdots(\alpha-n+1)}{n!}x^n+o(x^n)
\end{align}

\subsection{做题应用} \label{tylor-app}

其中,\textbf{奇函数}的麦克劳林级数展开中偶数阶导数系数为0。

\section{不等式} \label{inequlity}

\subsection{随听课新增的}

\begin{lemma}
    \[
        \dfrac{x}{1+x} < \ln (1+x) < x
    \]
\end{lemma}



\subsection{基本不等式}

考研数学中常用的不等式有以下几类:

对非负实数$a,b$,有
\begin{equation}
	a+b \geq 2\sqrt {ab}
\end{equation}
等号成立当且仅当$a=b$.

对正实数$a,b$,有
\begin{equation}
	\dfrac{a}{b}+\dfrac{b}{a} \geq 2
\end{equation}
等号成立当且仅当$a=b$.

对任意实数$x,y$,有
\begin{equation}
	(x+y)^2 \leq 2(x^2+y^2)
\end{equation}
等号成立当且仅当$x=y$或$x=-y$.

对任意实数$x,y,z$,有
\begin{equation}
	(x+y+z)^2 \leq 3(x^2+y^2+z^2)
\end{equation}
等号成立当且仅当$x=y=z$.

对任意实数$x_1,x_2,\dots,x_n$和正整数$n$,有
\begin{equation}
	\dfrac{x_1+x_2+\cdots+x_n}{n} \geq \sqrt[n]{x_1x_2\cdots x_n}
\end{equation}
等号成立当且仅当$x_1=x_2=\cdots=x_n$.

对任意实数$a,b,c$和正整数$n$,有
\begin{equation}
	(a^n+b^n+c^n)^3 \geq 27a^nb^nc^n(a+b+c)^n
\end{equation}
等号成立当且仅当$a=b=c$.

对任意实数$a,b,c,d$和正整数$n,m$,有
\begin{equation}
	\begin{array}{c}
		(a^n+b^n+c^n+d^n)(a^m+b^m+c^m+d^m)\\ \geq \\(a^{n+m}+b^{n+m}+c^{n+m}+d^{n+m})^2
	\end{array}
\end{equation}
等号成立当且仅当$a=b=c=d$.

\subsection{三角不等式}
考研数学中常用的三角不等式有以下几种:

\subsubsection{绝对值形式}
对任意实数$a,b$,有
\begin{equation}
	|a-b| \leq |a|+|b|
\end{equation}
等号成立当且仅当$a,b$同号或其中一个为零.

对任意实数$a,b,c$,有
\begin{equation}
	|a+b+c| \leq |a|+|b|+|c|
\end{equation}
等号成立当且仅当$a,b,c$同号或其中两个为零.

对任意实数$a_1,a_2,\dots,a_n$和正整数$n$,有
\begin{equation}
	|a_1+a_2+\cdots+a_n| \leq |a_1|+|a_2|+\cdots+|a_n|
\end{equation}
等号成立当且仅当$a_1,a_2,\dots,a_n$同号或其中$n-1$个为零.

\subsubsection{向量形式}
对$n$维向量 
$\pmb{x}=(x_1,x_2,\dots,x_n),$
$\pmb{y}=(y_1,y_2,\dots,y_n)$,有
\begin{equation}
	|\pmb{x}|-|\pmb{y}|\leqslant |\pmb{x}\pm\pmb{y}|\leqslant |\pmb{x}|+|\pmb{y}|
\end{equation}
等号成立当且仅当$\pmb{x}\parallel\pmb{y}$.

对$n$维向量$\pmb{x}=(x_1,x_2,\dots,x_n),\pmb{y}=(y_1,y_2,\dots,y_n),$
$\pmb{z}=(z_1,z_2,\dots,z_n)$,有
\begin{equation}
	|\pmb{x}+\pmb{y}+\pmb{z}| \leq |\pmb{x}|+|\pmb{y}|+|\pmb{z}|
\end{equation}

\section{巨算符} \label{giant-operator}

\cite[page A36]{stewart}
\[
    \sum \dfrac{f(x)}{g(x)} \neq \dfrac{\sum f(x)}{\sum g(x)}
\]


\cleardoublepage
\chapter{极限}

求极限的过程会把一种极限类型转化为另一种。
请在实际做题的时候主义当前的极限类型,并选择最合适的做法,灵活运用。
使用导数定义求极限请参考下一章。

和定积分有关的极限问题请参见 \ref{limit-questions-involved-definite-integral}.

\section{通用技术}

本节将会对下文将会详细描述的各种求极限类型所用到的一些通用手段进行总结和归纳。
原则上这些内容应当背熟用熟练。

对于所有题型,这些方法都适合。但是对于选择填空题,要应用更多的应试策略。
比如带入选项,或者是通过题目条件举出特殊函数的例子等。

根据极限的定义可知
\[
    \lim_{x \to a^-} f(x) = \lim_{x \to a^+} f(x) = L \Rightarrow \lim_{x\to a} f(x) = L
\]
因此在处理某些初等函数的时候就不得不注意了
\begin{example}
    \cite[page 9]{yc}
    \[
        \lim_{x \to 1} \dfrac{x^2 - 1}{x-1} e^{\frac{1}{x-1}}
    \]
    式子中出现 $e^{\frac{1}{x-1}}$ 则需要注意分类讨论
    \begin{align*}
        \lim_{x \to 1^+} \dfrac{x^2 - 1}{x-1} e^{\frac{1}{x-1}} &= +\infty\\
        \lim_{x \to 1^-} \dfrac{x^2 - 1}{x-1} e^{\frac{1}{x-1}} &= 0
    \end{align*}
    则极限不存在,且不等于 $\infty$.
\end{example} 

\subsection{L'Hospital}

应用 L'Hospital 法则求极限的时候,
要注意极限中有关函数的可导性到多少阶有效。
$n$ 阶可导\textbf{不代表} $n$ 阶导函数存在,
也\textbf{不能保证}$n$阶导函数连续。
因此对有可导阶数有规定的情况,L'Hospital 只能应用到 $n-1$ 阶。

\subsection{选择题策略}

使用排除法的时候请务必将每一个选项都试一试,这样可以保证没有漏选。

\subsection{等价无穷小}  \label{super-small}
所有等价无穷小替换都可从 Maclaurin's series 推导出来。

等价无穷小可以在符合一定条件的加减项上使用,
可以在因子上使用。
但是不能在指数上使用(考研数学不建议,是否可用之判断比较复杂)。

\subsubsection{公式}
这部分等价无穷小可以通过他们的函数图像辅以记忆,
由于等价无穷小是在讨论无限接近于0点的函数状态,
因此要格外关注函数在0点附近性态。
\begin{lemma}[等价无穷小]
	\begin{align}
		x&\sim\sin{x}\sim\tan{x} \\ 
		&\sim\arcsin{x}\sim\arctan{x} \\
		&\sim\ln{1+x}\sim e^x-1
	\end{align}
\end{lemma}

还有一组常用公式是从 Maclaurin's series 推导而来的,作为\textbf{常用结论}使用:
\begin{lemma}[等价无穷小]
	\begin{align}
		x - \sin{x}     &\sim \frac{x^3}{6} \label{rep1} \\
		\tan{x} - x     &\sim \frac{x^3}{3} \label{rep2} \\
		x - \ln{1+x}    &\sim \frac{x^2}{2} \\
		\arcsin{x} - x  &\sim \frac{x^3}{6} \label{rep3}\\
		x - \arctan{x}  &\sim \frac{x^3}{3} \label{rep4}\\
		(1+x)^\alpha-1  &\sim \alpha x
	\end{align}
\end{lemma}
上述公式\eqref{rep3}和\eqref{rep4}可由\eqref{rep1}和\eqref{rep2}分别通过项的等价代换得来,
因为他们代换后两项(加减号两边)不等价。

\subsubsection{何时可用?}
除了我们熟悉的式子中的\textbf{因子可以直接使用}等价无穷小代换之外,
实际上因子中的\textbf{项}也可以。但是有条件:
\textbf{替换后两项不能等价\footnote{相等 $\Rightarrow$ 等价,反之不成立。}。}比如:
\[
\lim_{x \to 0} \frac{\sin{x} - \tan{x} }{x^3} \neq 0
\]
中LHS的分子中不可以将两个项替换成$x$,因为 $x$ 显然和 $x$ 等价.
换句话说,我们不让相加减的两个函数的 Tylor Expantion 中的第一项
(所谓的等价无穷小量)消去即可。即:
\begin{align}
	\lim_{ x\to x_0 } \frac{a_1}{a_2} &\neq 1  &\Rightarrow a_1 - a_2 \sim b_1 - b_2 \\
	\lim_{ x\to x_0 } \frac{a_1}{a_2} &\neq -1 &\Rightarrow a_1 + a_2 \sim b_1 + b_2
\end{align}

\subsubsection{定积分中的替换}
Suppose $f(x)$ and $g(x)$ are continuous on 某邻域,and $\lim_{x \to 0} \frac{f(x)}{g(x)} = 1$
Thus, 
\begin{equation}
	\int_0^{x} f(t) dt \sim \int_0^{x} g(t) dt
\end{equation}
如果$g(x)=c$也是可以使用本结论的,请不要忽视这一重要用途。For example:
\[
\int_0^x e^{t^2} dt \sim \int_0^{x} 1 dt
\]

还有\textbf{更一般}的情况:
\begin{lemma}
	Suppose $f(x)$ and $g(x)$ are continuous on 某邻域,and $\lim_{x \to a} \frac{f(x)}{g(x)} = 1$ 
	Thus, 
	\begin{equation}
		\int_a^{x} f(t) dt \sim \int_a^{x} g(t) dt
	\end{equation}
\end{lemma}

\subsection{Lagrange Mean Value Theorem} \label{lagrange-limit}
当极限式子中某部分出现两个函数相减的形式,就可使用这种方法化简。
其中析出的一个 $\xi$ 往往可以通过极限趋向来确定,作为非零因子\ref{non-zero-factor}算出。

当 $x\to0$的时候,如果析出的函数导数是 $e^{\xi} = 1$ 那么,不难理解有
\begin{equation}
	e^{\alpha(x)} - e^{\beta(x)} \sim \alpha(x) - \beta(x)
\end{equation}
其中 $\alpha(x) \to 0$ 且 $\beta(x) \to 0$,但是这个结论\textbf{考研不能直接使用}

\section{其他重要结论}
\begin{lemma}
	\begin{equation}
		\lim_{n \to \infty} \sqrt[n]{a^{n}_{1} + a^{n}_{2} + 
			\cdots + a^{n}_{n}} = \max_{1\leq i \leq n}{a_{i}^{n}}
	\end{equation}
\end{lemma}

\begin{lemma}
    \begin{equation}
        \frac{0}{()} \to c \Rightarrow () \to 0
    \end{equation}
    which $c$ is a constant number.
    同理
    \begin{equation}
        \frac{()}{0} \to c \Rightarrow () \to 0
    \end{equation}
    which $c$ is a constant number.
\end{lemma}

\begin{lemma}
    \[
        \lim_{x \to 0^+} x^{\alpha} \ln^{\beta}x = 0
    \]
    Once $a$ is a positive number. ??? %TODO
\end{lemma}

\section{极限之 Squeeze Theorem 还是 定积分定义?}
\label{use-squeeze-or-definition-of-integral}

观察数列求和式子中每一项的分母,找到其中的\textbf{变化部分和不变部分}。
\begin{equation} \label{eq:example-squeeze}
	\lim_{n \to 0} \left[ 
	\dfrac{n}{n^2+1} + \dfrac{n}{n^2+2} + \cdots + \dfrac{n}{n^2+n} 
	\right]  
\end{equation}
\begin{equation} \label{eq:example-defination-int}
	\lim_{n \to 0} \left[ 
	\dfrac{n}{n^2+1^2} + \dfrac{n}{n^2+2^2} + \cdots + \dfrac{n}{n^2+n^2} 
	\right]  
\end{equation}
本例子中\ref{eq:example-squeeze}各项分母中的两项里面
\textbf{不变的部分为$n^2$,变化的部分是$1,2,3,4,\cdots$}.
两部分具有\textbf{不同数量级},这种情况使用\textbf{Squeeze Theorem}。

本例子中\ref{eq:example-defination-int}各项分母中的两项里面
\textbf{不变的部分为$n^2$,变化的部分是$1^2,2^2,3^2,4^2,\cdots$}.
两部分具有\textbf{相同数量级},
这种情况使用\textbf{定积分定义}
(
    Definition\ref{defination-definite-integral}, 
    Equation\ref{eq:the-other-definition-definite-integral}
)。

那么针对使用定积分解决的问题,
我们接下来提一个因子
\footnote{
    被武忠祥老师称为“可爱因子”, 事实上由 
    $\Delta x$变形而来, 
    参见 Equation \ref{eq:the-other-definition-definite-integral}
},这样可构造出黎曼和定积分定义:
\begin{equation*}
	\lim_{n \to 0} \dfrac{1}{n} \left[ 
	\dfrac{1}{1+\left(\dfrac{1}{n}\right)^2} + 
	\dfrac{1}{1+\left(\dfrac{2}{n}\right)^2} + \cdots + 
	\dfrac{1}{1+\left(\dfrac{n}{n}\right)^2}
	\right]  
\end{equation*}
这样 $f(x) = \dfrac{1}{1+x^2}$
\begin{equation*}
	\mbox{LHS} = \int_{0}^{1} \dfrac{1}{1+x^2} dx = \pi / 4
\end{equation*}

\section{0/0型}
\subsection{常用方法}
\begin{enumerate}
	\item L'Hospital
	\item 等价无穷小代换 \ref{super-small}
	\item Tylor 公式
	\item 加减项(若上述方法均不方便)\footnote{武讲义0/0例3,三星笔记求极限通法2P10}
\end{enumerate}

这种类型的极限之所以不好求,
是因为分母有0因子。
则将分母上的0因子消掉,也是一个很好的办法。

加减项可参考本例:
\begin{align*}
	&\lim_{x \to 0} \dfrac{\arcsin{x} - \sin{x}}{\arctan{x} - \tan{x}} \\
	= &\lim_{x \to 0} \dfrac{(\arcsin{x} -x) - (\sin{x}-x)}{(\arctan{x}-x) - (\tan{x}-x)}
\end{align*}
然后使用无穷小替换\ref{super-small}快速求解

\subsection{原式化简}
\begin{enumerate}
	\item 非零因子(fator)先求出 \label{non-zero-factor}
	\item 有理化
	\item 变量代换
\end{enumerate}

其中,往往在有理化过程后出现非零因子,这个时候请立刻将非零因子求出,提到式子外面。比如:

\begin{align*}
	&\lim_{x \to 0} \dfrac{\sqrt{1+\tan{x}} - \sqrt{1+\sin{x}}}{x \ln{1+x} - x^2} \\
	=&\lim_{x \to 0} \dfrac{\tan{x} - \sin{x}}{x \ln{1+x} - x^2} \cdot
	\underbrace{\dfrac{1}{\sqrt{1+\tan{x}} + \sqrt{1+\sin{x}}}}_{\mbox{非零因子}}
\end{align*}

例题选自武忠祥强化课程,本题目接下来可以通过\textbf{拉格朗日中值定理}\ref{lagrange-limit}来求解。
详情请参考\,求极限通法2 三星笔记。

本题分子里面含有根号,实际上也就是开 $1/2$ 次方,这种情况还可以构造
$(1+x)^\alpha - a \sim \alpha x$
\ref{super-small}
\begin{equation*}
	\begin{array}{l}
		\mbox{LHS} = \lim_{x \to 0}
		\dfrac{\sqrt{1+\sin{x}}\left[ 
			\sqrt{1+\dfrac{\tan{x}-\sin{x}}{1+\sin{x}}}	- 1
			\right] }{x \ln{1+x} - x^2}
	\end{array}
\end{equation*}

\section{无穷比无穷}

\subsection{常用方法}

\begin{enumerate}
	\item L'Hospital
	\item “抓大头”\footnote{大题必须要写好步骤,必须要体现分母无穷大于分子}
\end{enumerate}

\section{1 to Infinity}
\subsection{常用方法}
\begin{enumerate}
	\item 凑基本极限
	\item 改写成指数
	\item 应用常用结论
\end{enumerate}

其中常用结论可以精炼为下面三步:
\begin{enumerate}
	\item 写标准型,令
	\[\mbox{LHS} = \lim{} \left[ 1+\alpha(x)\right] ^{\beta(x)}\]
	其中 \[\alpha(x)\to0, \beta(x)\to\infty\]
	\item 求极限 $A = \lim \alpha(x)\beta(x)$
	\item 写结果 $\mbox{LHS} = e^{A}$
\end{enumerate}

\section{确定求极限式子里的参数}

这种题目就是要将其按照正常极限进行求解,在求的过程中或者是最后得到要求的参数。

\begin{lemma} \label{lm:inf-zero-zero}
	\begin{equation}
		\infty \cdot () \to 0 \Rightarrow () \to 0
	\end{equation}

\end{lemma}

例题:
\[
\lim_{x \to -\infty} \sqrt{x^2+x+1} +  ax + b = 0
\]
求 $a, b.$
这道题我们可以首先提出一个 $x$,这样就可以构造 \ref{lm:inf-zero-zero}:

\begin{align*}
	&\mbox{LHS} = \lim_{x \to -\infty} \left(\overbrace{-x}^{ \to \infty }\right)
	\left( 
	\underbrace{\sqrt{1+\dfrac{1}{x}+\dfrac{1}{x^2}} + a + \dfrac{b}{x}}_{\to 0}
	\right) = 0\\
	&\therefore 1-a = 0, a = 1.
\end{align*}
进一步就可以轻松求出 $b$.

\section{无穷小阶数比较}
三种主要方法:
\begin{enumerate}
	\item L'Hospital
	\item 等价无穷小代换
	\item Tylor 公式
\end{enumerate}

这三种方法都是用来在构造完 $f(x)/g(x)$ 之后进行约分化简的时候使用的。
考察这种知识点的提醒一般比较集中在选择题。
因此在做题的时候要格外注意策略。
所谓的定义法也就是构造
\[
    \lim_{x\to 0} \dfrac{f(x)}{g(x)}
\]
计算这个极限,看结果是 $\infty, 0$还是$c$。 

\begin{lemma}[L'Hospital求导定阶]
	若 $x \to 0$ 时 $f(x)$ 是无穷小量,且 $f'(x)$ 是 $x$ 的 $k(k>0)$ 阶无穷小,
	则 $f(x)$ 是 $k+1$ 阶无穷小。即
	\begin{equation}
		\lim_{x \to 0} \dfrac{f(x)}{x^{n+1}} = \lim_{x \to 0} \dfrac{f'(x)}{(n+1)x^{n}} \neq 0
	\end{equation}
\end{lemma}

\begin{lemma}
	若 $f(x)$ 在 $x=0$ 的某邻域内连续,而且当 $x\to0$ 时 $f(x)$ 是 $x$的 $m$ 阶无穷小,
	$\sigma(x)$ 是 $x$ 的 $n$ 阶无穷小,则当 $x\to0$ 时
	\begin{equation*}
		F(x) = \int_{0}^{\sigma(x)} f(t) dt
	\end{equation*}
	是 $x$ 的 $n(m+1)$ 阶无穷小
\end{lemma}

\begin{lemma}
	低阶 + 高阶 $\sim$ 低阶
	
	$x \to 0, x^2+x^3=x^2$
\end{lemma}

\section{连续定义}

\begin{definition}
	函数某点的极限值和函数值相等,称该函数在该点连续。
\end{definition}
当然也有左右连续,将上述定义中的极限值分别换成左/右极限值,则左右连续的定义易知。

\section{间断点分类问题}
一般这种题型要求\textbf{讨论}间断点的类型,则需要写下明确的间断点的位置。
并且对每个间断点的类型进行讨论,这种讨论必须要对间断点两边分别求极限才可行。
或者那种比较容易判断的\textbf{可去}间断点,则直接求极限。

\begin{itemize}
    \item 第一类间断点
        \begin{itemize} 
            \item 可去间断点
            \item 跳跃间断点
        \end{itemize}
    \item 第二类间断点
        \begin{itemize}
            \item 函数至少有一侧极限不存在的点
        \end{itemize}
\end{itemize}


\cleardoublepage
\twocolumn
\chapter{一元函数微分学}

\begin{definition}
\begin{align}
	f'(x_0) &= \lim_{\Delta x \to 0} \dfrac{f(x_0+\Delta x) - f(x_0)}{\Delta x} \\
	        &= \lim_{x \to x_0} \dfrac{f(x) - f(x_0)}{x - x_0}
\end{align}
\end{definition}

\section{通用技术}

$n$ 阶可导不能保证 $n$ 阶导函数有极限,也不能保证 $n$ 阶导函数连续。

相当一部分题目是和参数有关的,
当参数作为自变量的系数的时候对其的讨论就会变得复杂。
可以先将式子出以参数的所在的因子的另一部分,是的参数单独出来。
请参考 \ref{number-of-roots-question}

可导 $\Rightarrow$ 连续,相关求参数的题目可以用这一性质求出,
比如 \cite[quest27]{w660}

$f^{(n)}$ 在 $x = a$ 存在,不代表 $f^{(n)}$ 在 $a$ 的某邻域一定有定义。
这个时候就不能应用L'hospital 法则。

有关梯度,请参考小节\ref{gradient}.

\subsection{周期函数和导数}

周期函数的导数仍然是周期函数,
若该周期函数具有\textbf{奇偶性},
则它的导数的\textbf{奇偶性刚好相反}。

参见 \cite[quest29]{w660}

\section{利用导数定义求极限}

因为导数就是用极限来定义的,因此如果能将要求的极限式子变化为导数的定义形式
则可以应用求导的方法在求极限上。题目无非就是凑导数定义式。

比如
\[
	f(-1) = 1, f'(-1) = 2, \mbox{求} \lim_{x \to 1 } \dfrac{f(2-3x) - 1}{x-1}
\]
首先观察,分子出现了一个 $-1$,恰好就是 $f(-1)$ ,分母不符合要求,但我们可以乘上一个 factor
凑出我们想要的式子:
\begin{align*}
	\mbox{LHS} &= \lim_{x \to 1 } \dfrac{f(-1+\overbrace{3(1-x)}^{\Delta \to 0}) -
		 f(-1)}{\underbrace{3(1-x)}_{\Delta \to 0}} \cdot \dfrac{3(1-x)}{x-1} \\
		 &= f'(-1) \cdot \lim_{x \to 1} \dfrac{3(1-x)}{x-1} = -6
\end{align*}
本题还可以按照题目给的条件:$f(-1) = 1, f'(-1) = 2$ 直接给出一个满足条件的简单函数并带入求值,
比如 $f(x) = 2x+3$。
适合选择填空。
注意上面的 $\Delta$ 只要能够 $\to 0$ 就可以。

\section{利用导数定义求导数}

略。

\section{利用导数定义判断函数可导性}

略。

\section{导数的几何意义}

略。

\subsection{求切线}

求切线就注意一点,函数值和该点导数都应该相同,不能仅导数相同。

\section{复合函数求导法}

\subsection{函数奇偶性辅助解题}

比如
\[
    f(x) = \ln(x+\sqrt{1+x^2})
\]
求 $f''(x)$.
这种题目直接使用符合函数求导法则不是很方便,
在做题之前可先观察函数类型,可以发现本函数
是\textbf{奇函数},因此麦克劳林公式展开中,没有偶次导数项。
因此二阶导数为 0.

\subsection{链式法则以外的方法}

另外,复合函数求导使用链式法则需要两个函数的导数都存在。
若内层函数在某点的导数不存在,\textbf{并不能说明整体导数不存在},
因此,若链式法则无效,则应当尝试将复合函数解析式子求出,
在进行下一步求值。

另外,确定分段点要使用导数定义判断导数是否存在,注意定义域。

\subsection{只能用链式求导法则的情况}

比如 设
\[
    \phi (x) = 
    \left\{
        \begin{array}{rl}
            x^3 \sin \frac{1}{x} &, x \neq 0 \\
            0  &, x = 0
        \end{array}
    \right.
\]
函数 $f(x)$ 可导,求$F(x)=f(g(x))$ 的导数.
这个题目,如果求出复合后的函数式子之后对其凑导数定义则不通。
原因:
\[
    F'(0) = \underbrace{\lim_{x \to 0} \dfrac{f(x^3 \sin \frac{1}{x}) - f(0)}{x^3 \sin \frac{1}{x}}}_{\mbox{D.N.E}}
    \cdot \lim_{x \to 0} \dfrac{x^3 \sin \frac{1}{x}}{x}
\]
不存在的原因是在极限的函数在 $x=0$ 的任何去心邻域内都有没定义的点 $x = 1/n\pi$ ($n$ 充分大).

\section{隐函数求导}

注意先对式子进行化简再进行隐函数求导。
另外,隐函数求导过程中可能会出现原表达式。在计算过程中可以利用这一点简化
计算步骤,写结果的时候注意化简。
参见 \cite[quest35]{w660}.

\section{分段函数求导}

此类问题最后都要观察,看看是否能对最终答案进行化简。

\subsection{段是区间和区间的}

即形如
\[
    f(x) = 
    \left\{
        \begin{array}{rl}
            g(x), &x \leq x_0 \\
            k(x), &x > x_0
        \end{array}
    \right. 
\]
的.

\textbf{分段点使用导数定义进行求导},非分段点使用公式。
求完了分段点位置的导数之后,如果原分段函数中存在自变量取值范围
为闭区间或者出现$\geq$;$\leq$ 的要\textbf{将分段点带入到
使用公式求得的导函数中确认是否和定义法求得的值相同,}
最后再写答案。

如果使用定义发现\textbf{分段点不可导},
只需要在写答案的时候不包括分段点即可。
题目可以参考 \cite[quest26]{w660}.

\subsection{分段是点和区间的}

即形如
\[
    f(x) = 
    \left\{
        \begin{array}{rl}
            g(x),&x \neq x_0 \\
            A, &x = x_0
        \end{array}
    \right. 
\]
的\footnote{\cite[page 16]{w660ans}}.

仍然先将段的导函数求出,
之后使用导数定义求那个点的导数.
参见 \cite[quest27]{w660}.

\section{参数方程求导}

\begin{lemma}
    \begin{equation}
        \dfrac{\mbox{d}y}{\mbox{d}x} = \dfrac{y'(t)}{x'(t)}
    \end{equation}

    \begin{equation}\label{eq:sec-ord-sub-der}
        \dfrac{\mbox{d}^2 y}{\mbox{d}x^2} = \dfrac{y''(t) x'(t) -x''(t) y'(t)}{x'^3(t)}
    \end{equation}

    等式 \ref{eq:sec-ord-sub-der} 比较适合求具体点的导数值使用,
    求n阶导函数问题可以:
    \begin{equation}
        \dfrac{\mbox{d}^2y}{\mbox{d}x^2} = 
        \dfrac{\mbox{d}}{\mbox{d}t} 
        \left(
            \dfrac{y'(t)}{x'(t)} 
        \right)
        \dfrac{1}{x'(t)}
    \end{equation}
\end{lemma}

\section{反函数求导法}

\begin{lemma}
    \begin{equation}
        \phi (y) = \dfrac{\mbox{d}x}{\mbox{d}y} = \dfrac{1}{\dfrac{\mbox{d}y}{\mbox{d}x}}
                   = \dfrac{1}{f'(x)}
    \end{equation}
\end{lemma}

\section{高阶导数}

方法:
\begin{enumerate}
    \item 代公式
    \item 求小阶数,归纳
    \item 使用Tylor公式,或者级数
\end{enumerate}
有关Tylor公式,对于求高阶导数的方法就是寻找n阶项的系数

另本题型中有理函数应当先进行拆分后分别求导。
包含三角函数的如果应用归纳法,则应当考虑每次求导后
先应用初等数学中三角函数相关简化步骤,再进一步求导。

\begin{lemma}
    \begin{equation}\label{eq:uv-n-order-derivitive}
        (uv)^{(n)} = \sum_{k=0}^{n} \mathbf{C}^{k}_{n} u^{(k)} v^{(n-k)}
    \end{equation}
    其中
    \begin{equation*}
        \mathbf{C}^{k}_{n} = \binom{n}{k} = \dfrac{n!}{k!(n-k)!}
    \end{equation*}
\end{lemma}

公式 \ref{eq:uv-n-order-derivitive} 用于计算 $f^{(n)}(c)$ 问题(\textbf{具体点}),
往往需要观察 $uv$ 两部分是否属于高阶导数为 0 的情况,
如果是,这个方法就很有用。

另外,还可以使用Tylor 展开来解决求某点的值的问题。
各典型函数的展开式参见 \ref{tylor}.
具体操作步骤请参考下面的例子:
\begin{example}
    求值:$f^{(5)}(0), f(x) = x^3 e^x$

    \[
        f(x) = f(0) + \cdots + \underbrace{\dfrac{f^{(5)}(0)}{5!}}_{\star} x^{5} + o(x^5)
    \]
    又
    \begin{align*}
        x^3 e^x &= x^3 \underbrace{(1+x+\dfrac{x^2}{2!} + o(x^2))}_{e^x\mbox{的Tylor展开}} \\
                &= x^3 + x^4 + \underbrace{\dfrac{1}{2!}}_{\star} x^5 + o(x^5)
    \end{align*}
    其中 
    \[
        \dfrac{f^{(5)}(0)}{5!} = \dfrac{1}{2!}
    \]
    所以 $f^{(5)}(0) = 60$
\end{example}

\section{导数应用}

Rolle's Theorm, Lagrange's Theorm 比较重要。
本节内容多是考察函数的性质,则可以先判断函数奇偶性判断函数的对称情况。

这类题型和函数性质(图像、单调性等)相关性很大,
在处理的时候要善于利用函数的性质,辅助解题。
利用函数性质辅助解题的有例子:
题目 
\begin{itemize}

    \item   \cite[page 70, pdf 81, 例1]{we} 和 
            \cite[page 75, pdf 86, 例6]{we}利用函数是偶函数,
            因此图像对称,从而你确定第二条渐近线。

    \item   题目 \cite[page 77, pdf 88, 例5, 例6]{we} 
            通过观察函数图像,辅助确定根的个数,
            有关根的个数请参阅 \ref{number-of-roots-question}。

\end{itemize}

\subsection{拐点}

请注意,写答案的时候,一定要使用序偶来表示拐点的位置,
而不能只用一个 $x=?$ 表示。

拐点就是函数凹凸性变化的点,也就是说当$x_0$ 使得
\[
    f''(x_0) = 0
\]
那么,点$(x_0, f(x_0))$ 就是函数 $f(x)$ 的拐点.
另外 若 $f'''(x_0) \neq 0$ 则可以判定 $(x_0, f(x_0))$ 是曲线的拐点。

有关详细知识点叙述请参见 \cite[page 69, pdf 80]{we}.

\subsection{判断根的个数和存在性}

方法:
\begin{itemize} 
    \item 存在性
    \begin{itemize}
        \item 零点定理
        \item 罗尔定理
    \end{itemize}

    \item 根的个数
    \begin{itemize}
        \item 单调性
        \item 罗尔定理推论
    \end{itemize}

    \item   针对上述方法都不好解决的或者是计算非常困难的题目,
            可以试一试特殊值\footnote{比如1,2,3 等比较像正确答案的数值}
            带入法。请参考 \cite[page 77, pdf 88, 例4]{we} 即
            \ref{ex:special-val-substitution}。

\end{itemize}

\begin{theorem} [Rolle's Theorem] \label{rolle-mean-value}
    Let $f$ be a function that satisfies the following three hypothesis:
    \begin{itemize}
        \item $f$ is continuous on the \textbf{closed} interval $[a, b]$.
        \item $f$ is differentiable on the \textbf{open} interval $(a, b)$.
        \item $f(a) = f(b)$.
    \end{itemize}
    then, 
    \[
        \exists \xi \in (a, b), f'(\xi) = 0.
    \]
    \cite[page 290, pdf 325]{stewart}.
\end{theorem}

\begin{theorem}[Cauchy Mean Value Theorem] \label{cauthy-mean-value}
    Let $f$ and $g$ be continuous functions on the closed interval $[a, b]$ and differentiable on the open interval $(a, b)$. If $g'(x) \neq 0$ for all $x \in (a, b)$, then there exists a point $c \in (a, b)$ such that
    \[
        \frac{f(b) - f(a)}{g(b) - g(a)} = \frac{f'(c)}{g'(c)}.
    \]
\end{theorem}

\subsubsection{根的存在性}

\begin{definition} {Existence of Roots Theorem (Root Existential Theorem)} 
    Every polynomial equation of odd degree with real coefficients has at least one real root.
\end{definition}
\begin{definition}{零点定理}
    设函数 $f(x)$ 在闭区间 $\left[a, b\right]$ 上连续,且 $f(a)$ 和 $f(b)$ 异号
    则在开区间 $\left(a, b\right)$ 则 $\exists \xi, f(\xi) = 0$
\end{definition}

若有问题 $F'(x) = f(x) = 0$ 如果 $F(x)$ 容易表示,并且存在 $F(a)=F(b)$,
且满足罗尔定理的其他要求,
则应用罗尔定理:
\[
    \exists \xi, F'(\xi) = 0
\]
原式得证。

\subsubsection{根的个数} \label{number-of-roots-question}

一个函数单调区间内至多有一个0点,不过别忘了求一下单调区间两个
端点的函数值。

\begin{lemma}{罗尔定理推论}
    若在\textbf{区间 $I$} 上 $f^{(n)}(x) \neq 0$ 
    则方程 $f(x)$ \textbf{最多有} $n$ 个实数根。
\end{lemma}
适罗尔定理推论适合求解至多有多少或者有且仅有多少根的问题。
后半部分就使用了罗尔定理的推论保证了至多有多少根.
至于区间 $I$ 是开是闭,主要看题干,提干是开那就开,是闭那就闭。

\begin{example}{\cite[page 77, pdf 88,例4]{we}}
    \label{ex:special-val-substitution}
    Prove that $f(x) = 2^x - x^2 - 1, f(x) = 0$ have exactly 3 real roots.
    \begin{proof}
        \[
            f(0) = 0, f(1) = 0, f(2) = -1 < 0, f(5) = 6 > 0
        \]
        Thus, there's at least single root in $[2, 5]$. 
        which means there're at least 3 roots in $\mathbb{R}$.
        Becasue $f'''(x) = 2^x \ln ^3 2 \neq 0, x \in \mathbb{R}$, there're \textbf{at most}
        3 roots.
        In other words, there're exactly 3 roots.
    \end{proof}
\end{example}

这种题型也比较喜欢和参数一起出,
这种情况应当先分离参数,比如\cite[page 77, pdf 88, 例5]{we}。
参数分离后往往对参数的讨论就会被延后,这样需要讨论的情况就会相对来说少很多。

进一步可参考:
\begin{itemize}
    \item \cite[page 85, pdf 96, example 6]{we}.
\end{itemize}

\subsection{(隐函数求)极值点}

可以参考 \cite[page 71, pdf 82, 例2]{we}.
此类问题可先对等式两边求导,之后 let $y' = 0$ 解出 $y$,
针对不同的 y 的解,结合题目要求进行验证排除不符合要求的选项。
之后再将符合要求的$y$ 带入原方程中, Thus, you will get an
equation which contains only one variable, namely $x$.
对于任何一种求极值点的问题,\textbf{都要应用其充分条件进行验证}。

其中,上文提到的各充分条件和必要条件请参考
\cite[page 68, pdf 79]{we}。

\subsection{证明函数不等式}

详情参阅 \cite[page 79, pdf 90]{we}

方法:
\begin{itemize}
    \item 单调性(最常用)
    \item 最大最小值
    \item Lagrange Mean Value Theorem
    \item Tylor Formula, 函数不等式往往考察函数的全局性质,从而需要使用拉格朗日余项。
    \item 函数的凹凸性
\end{itemize}

\subsection{微分中值定理的证明}

本部分内容难度比较大,应当结合\cite{we}进行复习。

题型:
\begin{itemize}
    \item $F\left[\xi, f'(\xi), f''(\xi)\right] = 0$.
        \begin{itemize}
            \item 分析法(还原)
            \item 微分方程法
            \item 常用辅助函数法
        \end{itemize}
    \item $F\left[\xi, \eta, f'(\xi), f'(\eta), f''(\xi), f''(\eta)\right] = 0$.
        \begin{itemize}
            \item 不要求 $\xi \neq \eta$
            \item 要求 $\xi \neq \eta$
        \end{itemize}
    \item $F\left[\xi, f^{(n)}(\xi)\right] \leq 0, (n \leq 2)$.
        \begin{itemize}
            \item Tylor\ref{tylor} 展开 with Lagrange 余项, 
                  $x_0$ 选择函数值导数值提供信息多的点.
        \end{itemize}
\end{itemize}

\subsubsection{第一类题型}

通过上述三种方法找到辅助函数然后使用罗尔定理推出
\[
    \exists \xi \in \left(a, b\right), F'(\xi) = 0.
\]
其中,$F'(\xi)$ 直接作为提干中给出条件的式子,
但是还有一种情况是作为题干中式子的一个 factor.
如果是后者,则应当对 $F'(x)$ 的解析式进行说明,
以便推导出 $F'(x) = 0 \Rightarrow \mbox{原式子成立}$.
例题请参见 \cite[page 82, pdf 93, example 2]{we} 等.

常见的辅助函数请参见 \cite[page 83, pdf 94]{we}.
其中有一条比较 general 的:
\begin{align*}
    \mbox{欲证}\quad f'(\xi) + g(\xi) f(\xi) = 0,  \\
    \mbox{let} \quad 
    F(x) = \exp \left\{\int g(x) \mbox{d}x\right\} f(x).
\end{align*}

\section{好题收录}

\begin{example}
    武忠祥每日一题278\footnote{\url{https://www.bilibili.com/video/BV1zj411z71E}}.
    \begin{enumerate}
        \item 证明:$(1-x^2) y^{(n+1)} - (2n+1) xy^{(n)} - n^2 y^{(n - 1)} = 0 (n \geq 1)$
        \item 求 $y^{(n)}(0)$
    \end{enumerate}
\end{example}

\begin{example} \label{general-app-of-deritative-example}
    函数 $y = \dfrac{(x-3)^2}{4(x-1)}$ 的单调增区间是\underline{\quad\quad}, 
    单调减区间是 \underline{\quad\quad}, 极值是 \underline{\quad\quad}, 凹凸区间是\underline{\quad\quad}.

    \begin{center}
    \raisebox{0.5ex}{\rule{\textwidth}{0.3pt}}
    \end{center}

    方法:
    \begin{enumerate}
        \item 对函数进行变形,便于后续求导
        \item 求导
        \item 画表
    \end{enumerate}
    \cite[page 21, question 43]{w660ans}
\end{example}
例题 \ref{general-app-of-deritative-example}
中的函数 $y$ 可以变形为
\[
    y = \dfrac{x-1}{4} - 1 + \dfrac{1}{x-1}
\]
则
\begin{align}
    y' &= \dfrac{1}{4} - \dfrac{1}{(x-1)^2} \label{eq:deritative-for-further-defierentiation}\\
       &= \dfrac{(x-3)(x+1)}{4(x-1)^2}      \label{eq:deritative-for-finding-critical-point}
\end{align}
上述两个式子都是 $y'$, 但是其中 \ref{eq:deritative-for-further-defierentiation} 适合下一步求二阶导,
\ref{eq:deritative-for-finding-critical-point} 适合找驻点。

下一步就是画表
\footnote{这种题目\textbf{不会出大题},不用顾虑怎么在答题纸上表达啊这个表},
在这里因为篇幅关系就不绘图了
\footnote{希望有看到的人能帮忙提PR}。
可以简单说一下这个表的样子,首先表头被三个驻点分隔,
三个驻点两侧和中间是以它们为界的区间。
之后分别在这些驻点和区间上求函数的增减性、
凹凸性和函数值\footnote{为了求极值}即可。




\cleardoublepage
\chapter{一元函数积分学}

本章考察的重点是
\textbf{定积分}\ref{finite-integral}和
\textbf{定积分的应用}\ref{app-finite-integral}.

\section{不定积分} \label{infinite-integral}

主要内容有:
\begin{itemize}
    \item 两个概念
        \begin{itemize}
            \item 原函数概念
            \item 不定积分概念
        \end{itemize}
    \item \textbf{三种积分方法(重点)}
        \begin{itemize}
            \item 第一类换元法
            \item 第二类换元法
            \item 分部积分法
        \end{itemize}
    \item 三种可积函数
        \begin{itemize}
            \item 有理函数积分
            \item 三角有理函数积分
            \item 含有无理函数的积分
        \end{itemize}
\end{itemize}
其中有关不定积分的计算,考试只会考察一些比较基本的题型。
难题可以仅简单了解一下。
有关难度练习,例题请参阅 \cite[page 96]{we} 开始的各例题。

\subsection{不容易记忆的公式}

\begin{align}
    &\int \dfrac{1}{{a^2 + x^2}} \, dx          &=& \dfrac{1}{a} \arctan \left( \dfrac{x}{a} \right)      &+ C\\
    &\int \dfrac{1}{{a^2 - x^2}} \, dx          &=& \dfrac{1}{2a} \ln \left| \dfrac{a + x}{a - x} \right| &+ C\\
    &\int \dfrac{1}{{\sqrt{a^2 + x^2}}} \, dx   &=& \ln \left| x + \sqrt{a^2 + x^2} \right|               &+ C\\
    &\int \dfrac{1}{{\sqrt{x^2 - a^2}}} \, dx   &=& \ln \left| x + \sqrt{x^2 - a^2} \right|               &+ C
\end{align}
更多公式请参见 \cite[page 93, pdf 104]{we}.

\subsection{两个概念的应用}

比如 \cite[page 101, example 1]{we}:
\begin{example}
    Let $f'(e^x) = \sin x$, what is the $f(x)$?
\end{example}
本题目一方面可以通过令 $t = e^x$ 的方法然后进行比较常规的做法解出。
另外还可以对等式两边同时对 $e^x$ 积,即:
\[
    \int f'(e^x) \mbox{d} e^x = \int \sin x \mbox{d} e^x
\]
则 $\mbox{LHS} = f(e^x)$\footnote{应用了两个概念的某一个}, 
$\mbox{RHS}$ 为一种很好的分部积分的形式。
进一步解得 $f(e^x) = (e^x/2) (\sin x - \cos x) + C$ 后,
再进行变量代换 $t = e^x$ 求出 $f(x)$ 即可。

\subsection{分部积分}

\begin{definition}
    Suppose both $u(x)$ and $v(x)$ has their one order continuous deritative repectively.
    Thus, 
    \[
        \int u \mbox{d} v = uv - \int v \mbox{d} u
    \]
\end{definition}

分部积分适用于 integrand 为两个不同类型函数的乘积时。
其中 $uv$ 的选择关系到做题速度,相关信息请参见 \cite[page 95, pdf 106]{we}.

\subsection{第一类换元法}

有理被积函数尤其是分式的时候,可以考虑 \textbf{加项减项拆}
的策略来将式子分为多个式子相加减的形式。
请参阅\cite[page 98, pdf 109, example 7]{we}

\subsection{第二类换元法}

\begin{table}
    \centering
    \begin{tabular}{cc}
        \toprule
        若式子中出现 & 令 \\
        \midrule
        $\sqrt{a^2 - x^2}$ & $x = a \sin t$或$x = a \cos t$ \\
        $\sqrt{a^2 + x^2}$ & $x = a \tan t$ \\
        $\sqrt{x^2 - a^2}$ & $x = a \sec t$ \\
        \bottomrule
    \end{tabular}
    \caption{常用第二类换元法替换}
    \label{tab:useful-sec-type-substitutions}
\end{table}
第二类换元法常用替换请参见表格 \ref{tab:useful-sec-type-substitutions}.

\subsection{三类常见可积函数中不熟悉的考点}

\subsubsection{三角有理积分}

即形如
\[
    \int R(\sin x, \cos x) \mbox{d} x 
\]
的。

可以采用\textit{万能代换}法,令 $t = \tan (t/2)$, 则有
\begin{equation}
    \int R(\sin x, \cos x) \mbox{d} x 
    = \int R\left(\dfrac{2t}{1+t^2}, \dfrac{1-t^2}{1+t^2}\right) 
    \dfrac{2}{1+t^2} \mbox{d} x
\end{equation}
该方法虽然通用性很好,但是计算比较繁琐复杂
\footnote{若三角函数的次数为1,则可以考虑},
不到万不得已,不适用该方法。

\begin{table}
    \centering
    \begin{tabular}{ccc}
        \toprule
        若 & 令 \\
        \midrule
        $R(- \sin x,   \cos x) = -R(\sin x, \cos x)$ & $u = \cos x$ \\
        $R(  \sin x, - \cos x) = -R(\sin x, \cos x)$ & $u = \sin x$ \\
        $R(- \sin x, - \cos x) =  R(\sin x, \cos x)$ & $u = \tan x$ \\
        \bottomrule
    \end{tabular}
    \caption{常用三角有理式代换}
    \label{tab:useful-tri-rational-substitutions}
\end{table}
这种问题,最好先使用三角变形、换元、分部积分。
表\ref{tab:useful-tri-rational-substitutions}是常用的集中换元和其条件。

\subsubsection{简单无理积分} \label{simple-irrational-integral}

\begin{equation}
    \int R\left(x, \sqrt[n]{\dfrac{ax+b}{cx+d}}\right) \mathrm{d} x
\end{equation}
令 $\sqrt[n]{\dfrac{ax+b}{cx+d}}=t$ 将其化为有理函数进行计算。
\footnote{
    比\cite{we}更详细的讲解,请参见
    \url{https://zhuanlan.zhihu.com/p/413132173}.
}
\begin{align*}
    x &= \dfrac{d t^n-b}{a-c t^n} \\
    \mathrm{d} x &= \dfrac{n t^{n-1} (a d-b c)}{\left(a-c t^n\right)^2} \mathrm{d} t
\end{align*}
其中,若 $n = 1$
\begin{equation}
    \mathrm{d} x = \dfrac{ad-bc}{(a-ct)^2}
\end{equation}
若 $n = 2$
\begin{equation}
    \mathrm{d} x = \dfrac{2t(ad-bc)}{(a-ct^2)^2}
\end{equation}

\begin{example}
    求
    \[
        \int \dfrac{1}{x} \sqrt{\dfrac{x+1}{x}} \mathrm{d} x
    \].

    令 $t =\sqrt{\dfrac{x+1}{x}}$,则 $x = \dfrac{1}{t^2 - 1}$,
    $\mathrm{d}x = - \dfrac{2t}{(t^2 - 1)^2}$.
    则
    \begin{align*}
        \int \dfrac{1}{x} \sqrt{\dfrac{x+1}{x}} \mathrm{d} x 
        &=& 
        &\int (t^2 - 1) t \dfrac{2t}{(t^2-1)^2} \mathrm{d} t \\
        &=& 
        &-2 \int \dfrac{t^2}{(t^2-1)^2} \mathrm{d} t\\
        &=& 
        &-2 \int \left(1+\dfrac{1}{t^2-1}\right) \mathrm{d} t\\
        &=& 
        &-2\left(t + \dfrac{1}{2} \ln \left|\dfrac{t-1}{t+1}\right|\right) + \mathrm{C}
    \end{align*}
\end{example}

\subsection{分段函数的不定积分}

本题型考察的就是分段点的处理,只需保证积分后两个分段是连续的即可
\footnote{在考研数学中这种情况无需考生证明可导性,
可以证明连续分段被积函数若其两段原函数是连续能够推出原函数可导}。
参见 \cite[page 101, pdf 112, example 5]{we}.

\begin{example}
    求不定积分 $\int e ^{-|x|} \mathrm{d} x$

    \textbf{方法一:}
    \[
        e^{-|x|} = \left\{
            \begin{array}{rl}
                \mathrm{e} ^{-x} &, x \leq 0, \\
                \mathrm{e} ^x    &, x < 0.
            \end{array}
        \right.
    \]
    \[
        \int \mathrm{e} ^{-|x|} \mathrm{d} x = \left\{
            \begin{array}{rl}
                - \mathrm{e} ^{-x} + \mathrm{C_1} &, x \leq 0, \\
                  \mathrm{e} ^x    + \mathrm{C_2} &, x < 0.
            \end{array}
        \right.
    \]

    \uline{
        $\mathrm{e}^{-|x|}$ 连续,
        原函数$F(x)$\textbf{必连续}, 从而$F(x)$
        在 $x = 0$ \textbf{连续}
    },由于
    \begin{align*}
        &\lim_{x \to 0^+} F(x) &= &\lim_{x \to 0^+}(-e^{-x} + C_{1})&=& -1& +\mathrm{C_{1}}  \\
        &\lim_{x \to 0^-} F(x) &= &\lim_{x \to 0^-}(e^{x} + C_{2})  &=&  1& +\mathrm{C_{2}} 
    \end{align*}
    所以 $ -1 + \mathrm{C_1} = 1 + \mathrm{C_2}$.
    \textbf{令 $C_1 = C$,则 $C_2 = -2 + C$} 因此
    \[
        \int \mathrm{e}^{-|x|} = \left\{
            \begin{array}{rl}
                - \mathrm{e}^{-x} + \mathrm{C} &, x \geq 0, \\
                  \mathrm{e}^{ x} - 2 + \mathrm{C} &, x < 0.
            \end{array}
        \right.
    \]

    \begin{center}
    \raisebox{0.5ex}{\rule{\textwidth}{0.3pt}}
    \end{center}

    \textbf{方法二:}
    由于 $\mathrm{e}^{-|x|}$ 连续,则$F(x) = \int_{0}^{x} \mathrm{e}^{-|x|} \mathrm{d} t$
    是其一个原函数,又
    \begin{align*}
        F(x) &= \int_{0}^{x} e^{-|x|} \mathrm{d} t = 
        \left\{
            \begin{array}{rl}
                \int_{0}^{x} \mathrm{e}^t    \mathrm{d} t &, x < 0, \\
                \int_{0}^{x} \mathrm{e}^{-t} \mathrm{d} t &, x \geq 0.
            \end{array}
        \right. \\
        &= \left\{
            \begin{array}{rl}
                \mathrm{e}^x - 1 &, x < 0 \\
                1 - \mathrm{e}^{-x} &, x \leq 0.
            \end{array}
        \right.
    \end{align*}
    \underline{则 $\int \mathrm{e}^{-|x|} \mathrm{d} x = F(x) + \mathrm{C}$}.
\end{example}
本例\footnote{注意下划线标记内容的表述方式。}
中,后一种不需要使用极限语言说明连续性,

\subsection{其它事项}

根据定义\ref{defination-definite-integral}可知,下列为不等式而非等式
\[
    \int \dfrac{f(x)}{g(x)} \mathrm{d}x 
    \neq 
    \dfrac{\int f(x) \mathrm{d} x}{\int g(x) \mathrm{d} x}
\]
Sigma 的使用注意事项请参见 \ref{giant-operator} 节。

\subsection{好题汇编}

\begin{example}\label{two-in-one-integral}
    求 $I_1 = \int \cos ^4 x \mathrm{d} x$, $I_2 = \int \sin ^4 x \mathrm{d} x$.
    \begin{align*}
        I_1 + I_2 &= \int 
                    \left[
                        (\sin ^2 x + \cos ^2 x)^2 - \dfrac{1}{2} \sin ^2 2x
                    \right] \mathrm{d}x \\
                  &= \int 1 - \dfrac{1}{2} \sin ^2 2x \mathrm{d} x \quad \mbox{降幂很巧妙}\\
                  &= \int 1 - \dfrac{1}{4} (1-\cos 4x) \mathrm{d} x \\
                  &= \dfrac{3}{4} x + \dfrac{1}{16} \sin x + C_1
    \end{align*}
    \begin{align*}
        I_1 - I_2 &= \int \cos ^4 x - \sin ^4 x \mathrm{d}x \\
                  &= \int \cos 2x \mathrm{d}x \\
                  &= \dfrac{1}{2} \sin 2x +C_2
    \end{align*}
    所以
    \begin{align*}
        I_1 &= \dfrac{1}{2} \left[(I_1+I_2)+(I_1-I_2)\right]=\mbox{something}_1 \\ 
        I_2 &= \dfrac{1}{2} \left[(I_1+I_2)-(I_1-I_2)\right]=\mbox{something}_2
    \end{align*}
    \cite[page 26, question 57]{w660ans}
\end{example}
Example\ref{two-in-one-integral}巧妙利用积分的和等于和的积分这一性质,
将两个积分组合到一起。这样更便于使用初等数学中三角恒等变换的知识
辅助解题。

\section{定积分}\label{finite-integral}

\subsection{定积分的概念}

\begin{itemize}
    \item 定积分概念
    \item 定积分几何意义
    \item 可积性
    \item \textbf{计算(重点)}
    \item \textbf{变上限积分(重点)}
    \item 定积分性质
    \item \underline{积分不等式(难点)}
\end{itemize}

使用黎曼和定义,根据几何意义的细微差别解析式形式也有细微不同,
但他们的值都是相同的:
\begin{definition}[Definite Integral]\label{defination-definite-integral}
    If $f$ is a fuction defined for $a \leq x \leq b$, 
    we devide the interval $[a, b]$ into $n$ subintervals of equal width
    $\Delta x = (b - a) / n$.
    We let $x_0 (=a), x_1, x_2, x_3, \cdots, x_n (=b)$ be the endpoints
    of these subintervals and we let $x_1^*, x_2^*, \cdots, x_n^*$ 
    be any \textbf{sample points} in these subintervals, so $x_u^*$
    lies in the $i$th subinterval $[x_{i-1}, x_i]$.
    Then the \textbf{definite integral of $f$ from $a$ to $b$} is
    \[
        \int_{a}^{b} f(x) \mbox{d} x = 
        \lim_{n \to \infty} \sum_{i = 1}^{n} f\left(x_i^*\right) \Delta x
    \]
    provided that this \emph{limit exists} and gives the same value for all
    possible choices of sample point. 
    \emph{If it does exist, we say that $f$ is \textbf{integrable} on $[a, b]$}.
    \cite[page 384]{stewart}
\end{definition}
上述定义式中 $\Delta x$ 可以写作 $\dfrac{1}{n}$
\footnote{被武忠祥老师称为“可爱因子”}, 
$x^*_i$ 可以写作 $\dfrac{i}{n}$, 
这样式子就变成了
\begin{equation}\label{eq:the-other-definition-definite-integral}
    \lim_{n \to \infty} \dfrac{1}{n}
    \sum_{i = 1}^{n} \underbrace{f\left(\dfrac{i}{n}\right)}_{\mbox{右端点函数值}} 
\end{equation}
\begin{equation}
    \lim_{n \to \infty} \dfrac{1}{n}
    \sum_{i = 1}^{n} \underbrace{f\left(\dfrac{i-1}{n}\right)}_{\mbox{左端点函数值}}
\end{equation}
\begin{equation}
    \lim_{n \to \infty} \dfrac{1}{n}
    \sum_{i = 1}^{n} \underbrace{f\left(\dfrac{2i-1}{2n}\right)}_{\mbox{中点函数值}} 
\end{equation}
上述三个公式是完全等价的。

\subsection{存在性}
\begin{align}
    f(x) \mbox{可积} &\quad         \Longrightarrow  \quad f(x) \mbox{有界} \\
    f(x) \mbox{有界} &\quad \cancel{\Longrightarrow} \quad f(x) \mbox{可积} \notag
\end{align}

\begin{theorem}[Integrable Theorem] \label{integrable-therom}
    If $f$ \underline{is continuous on $[a,b]$},
    or if $f$ has only 
    \underline{a finite number of \textit{jump discontinuities}}
    \footnote{按照武忠祥老师的说法,这里第一类间断点也可(跳跃、可去)}, 
    then $f$ is \emph{integrable} on $[a, b]$;
    that is, the definite integral $\int_a^b f(x) \mathrm{d} x$ exists.
\end{theorem}

\begin{align}
    f(x) \mbox{连续} &\quad         \Longrightarrow  \quad f(x) \mbox{可积} \\
    f(x) \mbox{可积} &\quad \cancel{\Longrightarrow} \quad f(x) \mbox{连续} \notag
\end{align}

\subsection{定积分计算}

\subsubsection{分部积分}

\begin{definition}[定积分分部积分法]
    设函数 $u(x)$ 和 $v(x)$ 在 $[a, b]$ 上有\textbf{连续一阶导数},则
    \[
        \int_a^b u \mbox{d} v = uv \left.\right|^{b}_{a} - \int_a^b v \mbox{d} u
    \]
\end{definition}

\subsubsection{公式}

周期性定积分公式:
\begin{lemma}
    \begin{equation}
        \int_a^{a + T} f(x) \mathrm{d} x = \int_0^T f(x) \mathrm{d} x
    \end{equation}
\end{lemma}

三角函数有关的公式:
\begin{lemma}
    \begin{multline}
        \int_0^{\frac{\pi}{2}} \sin ^n x \mathrm{d} x = \int_0^{\frac{\pi}{2}} \cos ^n x \mathrm{d} x \\
        = 
        \left\{ 
            \begin{array}{rl}
                \dfrac{n - 1}{n} \cdot \dfrac{n - 3}{n - 2} \cdots \dfrac{1}{2} \cdot \dfrac{\pi}{2} &, n \mbox{\ is even.}   \\[1em]
                \dfrac{n - 1}{n} \cdot \dfrac{n - 3}{n - 2} \cdots \dfrac{2}{3}                      &, n \mbox{\ is odd} > 1 \\
            \end{array}
        \right.
    \end{multline}
    
    若 $f(x)$ 连续,则
    \begin{equation}
        \int_0^{\pi} x f(\sin x) \mathrm{d} x = \dfrac{\pi}{2} \int_0^{\pi} f(\sin x) \mathrm{d} x
    \end{equation}
\end{lemma}

\section{广义积分中值定理} 

\begin{theorem}[广义积分中值定理] \label{general-mean-value-theorem-of-integral}
    若 $f(x), g(x)$ 在 $[a, b]$ 上连续,且 $g(x)$ 不变号,则
    \[
        \int_a^b f(x) g(x) \mathrm{d} x = f(\xi) \int_a^b g(x) \mathrm{d} x, a \leq \xi \leq b.
    \]
    当取 $g(x) = 1$ 则有
    \[
        \int_a^b f(x) \mathrm{d} x = f(\xi)(b - a), a < \xi < b.
    \]
    后者可直接用拉格朗日中值定理证明。
\end{theorem}
就某些问题,使用\emph{广义积分中值定理}比较方便。

\section{定积分的应用}\label{app-finite-integral}

利用定积分定义求极限请参见 Section \ref{use-squeeze-or-definition-of-integral}.

\cleardoublepage
\bibliography{ref}

\end{document}
