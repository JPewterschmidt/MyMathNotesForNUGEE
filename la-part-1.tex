\chapter{线性代数第一部分简明笔记}
\label{determinant}

\section{行列式}

\subsection{和块状矩阵有关的}

若 $A$ 和 $D$ 可逆, 则
\[
    \begin{vmatrix*}
        A & B \\
        0 & D
    \end{vmatrix*}
    =
    \abs*{A} \abs*{D}
\]

若 $A$ 可逆
\[
    \begin{vmatrix*}
        A & B \\
        C & D
    \end{vmatrix*}
    = \abs*{A} \abs*{D - CA^{-1} B}
\]

若 $D$ 可逆
\[
    \begin{vmatrix*}
        A & B \\
        C & D
    \end{vmatrix*}
    = \abs*{D} \abs*{A - BD^{-1} C}
\]

\section{Rank}
\label{rank}

\begin{corollary}
    下面是一些有关秩的性质: 

    \begin{itemize}
        \item $r(A) = r(A^T); \, r(A^TA) = r(A)$;
        \item 当 $k \neq 0$, $r(kA) = r(A)$;
        \item $r(A+B) \leq r(A) + r(B)$;
        \item $r(AB) \leq \min(r(A), r(B))$;
        \item 若$A$ 可逆,则 $r(AB) = r(BA) = r(B)$;
        \item 若$A$ 列满秩,则 $r(AB) = r(B)$;
        \item 若$A$ 是 $m \times n$ 矩阵,$B$ 是 $n \times s$ 矩阵,
            $AB = O$,则 $r(A) + r(B) \leq n$;
        \item \[
                r\left(\begin{bmatrix}
                        A & O \\
                        O & B 
                \end{bmatrix}\right) = r(A) + r(B)
            \]
        \item 若 $A \sim B$,则 $r(A) = r(B), r(A + kE) = r(B + kE)$.
    \end{itemize}
\end{corollary}

\begin{example}
    \[
        A = \begin{bmatrix}
            1 & 2 & 3 & 4 \\
            2 & 3 & 4 & 5 \\
            3 & 4 & 5 & 6 \\
            4 & 5 & 6 & 7 
        \end{bmatrix}, 
        B = \begin{bmatrix}
            0 & 1  & -1 &  2 \\
            0 & -1 &  2 &  3 \\
            0 & 0  & 1  & 4 \\
            0 & 0  & 0  & 2
        \end{bmatrix}
    \]
    求 $r(AB + 2A)$

    \cite[question 314]{w660}.

    \begin{gather*}
        AB + 2A = A(B + 2E) \\ 
        B + 2E = \begin{bmatrix}
            2 & 1 & -1 & 2 \\ 
            0 & 1 & 2  & 3 \\
            0 & 0 & 3  & 4 \\
            0 & 0 & 0  & 4 
        \end{bmatrix} \quad (\mbox{\small invertible})
    \end{gather*}

    矩阵 $B + 2E$ 可逆,故
    \[
        r(AB + 2A) = r(A(B + 2E)) = r(A)
    \]
    经初等变换后,矩阵的 rank 不变,有
    \begin{align*}
        A &= \begin{bmatrix}
            1 & 2 & 3 & 4 \\
            2 & 3 & 4 & 5 \\
            3 & 4 & 5 & 6 \\
            4 & 5 & 6 & 7 
        \end{bmatrix}
        \rightarrow 
        \begin{bmatrix}
            1 & 2 & 3 & 4 \\
            1 & 1 & 1 & 1 \\
            1 & 1 & 1 & 1 \\
            1 & 1 & 1 & 1 
        \end{bmatrix} \\
          &\rightarrow 
        \begin{bmatrix}
            1 & 2  & 3  & 4 \\
            0 & -1 & -2 & -3 \\
            0 & 0  & 0  & 0 \\
            0 & 0  & 0  & 0
        \end{bmatrix}
    \end{align*}

    所以 $r(AB + 2A) = 2$.
\end{example}
