\chapter{线性代数第一部分简明笔记}

\section{不同表示方法对照}

\begin{itemize}
    \item 基础解系 : free columns\cite[page 136]{strang}.
    \item $(a, b, c)^T$ 为列向量 : $(a, b ,c)$ is a column vector.
\end{itemize}

\section{行列式}
\label{determinant}

\subsection{和块状矩阵有关的}

若 $A$ 和 $D$ 可逆, 则
\[
    \begin{vmatrix*}
        A & B \\
        0 & D
    \end{vmatrix*}
    =
    \abs*{A} \abs*{D}
\]

若 $A$ 可逆
\[
    \begin{vmatrix*}
        A & B \\
        C & D
    \end{vmatrix*}
    = \abs*{A} \abs*{D - CA^{-1} B}
\]

若 $D$ 可逆
\[
    \begin{vmatrix*}
        A & B \\
        C & D
    \end{vmatrix*}
    = \abs*{D} \abs*{A - BD^{-1} C}
\]

\section{Rank}
\label{rank}

\begin{corollary}
    下面是一些有关秩的性质: 

    \begin{itemize}
        \item $r(A) = r(A^T); \, r(A^TA) = r(A)$;
        \item 当 $k \neq 0$, $r(kA) = r(A)$;
        \item $r(A+B) \leq r(A) + r(B)$;
        \item $r(AB) \leq \min(r(A), r(B))$;
        \item \textbf{若$A$ 可逆,则 $r(AB) = r(BA) = r(B)$;}
        \item \textbf{若$A$ 列满秩,则 $r(AB) = r(B)$;}
        \item \textbf{若$A$ 是 $m \times n$ 矩阵,$B$ 是 $n \times s$ 矩阵,
            $AB = O$,则 $r(A) + r(B) \leq n$;}
        \item \[
                r\left(\begin{bmatrix}
                        A & O \\
                        O & B 
                \end{bmatrix}\right) = r(A) + r(B)
            \]
        \item \textbf{若 $A \sim B$,则 $r(A) = r(B), r(A + kE) = r(B + kE)$.}
    \end{itemize}
\end{corollary}

\begin{example}
    \[
        A = \begin{bmatrix}
            1 & 2 & 3 & 4 \\
            2 & 3 & 4 & 5 \\
            3 & 4 & 5 & 6 \\
            4 & 5 & 6 & 7 
        \end{bmatrix}, 
        B = \begin{bmatrix}
            0 & 1  & -1 &  2 \\
            0 & -1 &  2 &  3 \\
            0 & 0  & 1  & 4 \\
            0 & 0  & 0  & 2
        \end{bmatrix}
    \]
    求 $r(AB + 2A)$

    \cite[question 314]{w660}.

    \begin{gather*}
        AB + 2A = A(B + 2E) \\ 
        B + 2E = \begin{bmatrix}
            2 & 1 & -1 & 2 \\ 
            0 & 1 & 2  & 3 \\
            0 & 0 & 3  & 4 \\
            0 & 0 & 0  & 4 
        \end{bmatrix} \quad (\mbox{\small invertible})
    \end{gather*}

    矩阵 $B + 2E$ 可逆,故
    \[
        r(AB + 2A) = r(A(B + 2E)) = r(A)
    \]
    经初等变换后,矩阵的 rank 不变,有
    \begin{align*}
        A &= \begin{bmatrix}
            1 & 2 & 3 & 4 \\
            2 & 3 & 4 & 5 \\
            3 & 4 & 5 & 6 \\
            4 & 5 & 6 & 7 
        \end{bmatrix}
        \rightarrow 
        \begin{bmatrix}
            1 & 2 & 3 & 4 \\
            1 & 1 & 1 & 1 \\
            1 & 1 & 1 & 1 \\
            1 & 1 & 1 & 1 
        \end{bmatrix} \\
          &\rightarrow 
        \begin{bmatrix}
            1 & 2  & 3  & 4 \\
            0 & -1 & -2 & -3 \\
            0 & 0  & 0  & 0 \\
            0 & 0  & 0  & 0
        \end{bmatrix}
    \end{align*}

    所以 $r(AB + 2A) = 2$.
\end{example}

\section{第一部分简述}

系数矩阵 full rank 的时候,方程有唯一解。
因为有三个 pivots, 所以方程总是能用 Gauss 消元法
\footnote{
    还有一个 Gauss-Jordan 法,这个方法也是先做 elimination,
    当矩阵化为 Upper-triangular 的时候,高斯就停下了。继续把矩阵化 
    Reduced-row-echlon form 是 Jordan 的贡献。
    而国内课本不知为何并没有在介绍 RREF 的时候介绍一下 Jordan.
}
法求出。

满足方程\footnote{等号右边为 $0$ 的称为齐次方程} 
$\mathbf Ax = 0$ 的所有 $x$ 构成矩阵 $\mathbf A$ 的
Null-space,
对于非齐次方程,其通解的基本想法是 
\[
    \mathbf x = {\mathbf x}_{p} + \mathbf x_n
\]
换句话说就是 \textbf{通解 = \emph{一个}特解 + \emph{所有} Null-space 解}.
Null-sapce 的数
\footnote{
    这取决于方程组的 free columns(这些自由列构成基础解系) 的个数
}个 basis 的线性组合来表示 Null-space. 比如某题
\cite[page 156]{strang}
求出的通解:
\[
    x = 
    \underbrace{
        \begin{bmatrix}
            7 \\ 0 \\ -3 \\ 0
        \end{bmatrix} 
    }_{\mathclap{\mbox{\small particular solution}}}
    + 
    \overbrace{
        c_1 \begin{bmatrix}
            -2 \\ 1 \\ 0 \\ 0
        \end{bmatrix}
        + c_2 \begin{bmatrix}
            4 \\ 0 \\ -4 \\ 1
        \end{bmatrix}
    }^{\mathclap{\mbox{\small the combination of the basis of Null-space}}}
\]
通过通解可以了解到该方程组有两个 free columns,
\emph{那么系数矩阵的 rank 就是}
\[
    r(A) = n - 2
\]
其中 $n$ 是该矩阵的行/列数。

根据矩阵乘法的数种理解方式 \cite[Section 2.4]{strang} 可知。
若方程组有解,那么等号右端在系数矩阵的 Column-sapce
\footnote{列空间,由该矩阵的列的线性组合 span.}
中。
所以
\begin{theorem}
    一个方程组有解,当且仅当等号右边在该方程组系数矩阵的 Column sapce 中。
\end{theorem}
Column sapce 的维度,取决于其各 basis 的最大维度。
在系数矩阵的视角上看这基本上就是矩阵的 rank.
上文提到了在 full rank 的情况下方程组必然有解,
反之则不一定。
一种情况是有无穷解,一种情况是无解。
有无穷解的情况正如上例所示,
无穷可以体现在对 $c_1$ 和 $c_2$ 的选取有无穷多种方案,
也可以体现在 Null sapce\footnote{本例中,Null space 是四维空间中的一个二维平面。} 作为一个 space 中有无穷个向量上。

快速判断方程组无解,可以将 augmented matrix 化为 RREF
\footnote{Reduced-row-echlon Form},若原系数矩阵中化出一全零行,
而 augmented 列对应位置却非零,那么很显然,
0的任何线性组合之结果都不可能为非0.
这时,方程组无解。

在实践中由于各种原因(比如测量误差、数据拟合等问题和需求),
方程组难以保证一定有解。
这种情况可以使用将该矩阵\emph{投影}到\emph{最近}的,
一个空间中。该\emph{投影}操作要务必保证原矩阵在满足实践需要的情况下其性质损失\emph{最少}。
和投影相关的话题请参见
\cite[Chapter 4.2]{strang}. 而 \cite[Chapter 4]{strang} 讲解的是正交。
正交对于投影操作来说非常重要,正交为上文中的某些字眼比如
\emph{最近,损失最小}提供了理论保证。

除了刚才提到的 Null sapce 和 Column space,
之外还有 Row sapce 和 Left null sapace.
Elimination 操作会改变 Column space 但不会改变 Row space。
因为 Elimination 操作是对各行的现行组合。如果两个矩阵的 Row space 相同,
则称他们为等价。用矩阵语言表述:
\[
    \mathbf{PAE} = \mathbf{B}
\]
其中 $P$ 为 Permutation 矩阵,应用在 $A$ 左侧会变换 $A$ 的行次序,
这也是一种线性组合。$E$ 为执行 Elimination 操作的矩阵。

