\chapter{全局通用技术}

\begin{lemma}
\begin{multline*}
	f, g \mbox{在} x_0 \mbox{有公共切线} \\ \Rightarrow f(x_0) = g(x_0), f'(x_0) = g'(x_0)
\end{multline*}
\end{lemma}
上述内容中小心遗漏 $f(x_0) = g(x_0)$

\begin{definition}
    \begin{equation}
        \mathbf{C}_{k}^{n}\binom{n}{k} = \frac{n!}{k!(n-k)!}
    \end{equation}
\end{definition}

\begin{lemma}
    \begin{align*}
        (a+b)^3 = a^3 + 3a^2b + 3ab^2 + b^3\\
        (a-b)^3 = a^3 - 3a^2b + 3ab^2 - b^3
    \end{align*}
\end{lemma}

\section{函数}

无论是导数还是积分,相关题目和函数性质(图像、单调性等)相关性总是很大,
在处理的时候要善于利用函数的性质,辅助解题。
利用函数性质辅助解题的有例子:
题目 
\begin{itemize}

    \item   \cite[page 70, pdf 81, 例1]{we} 和 
            \cite[page 75, pdf 86, 例6]{we}利用函数是偶函数,
            因此图像对称,从而你确定第二条渐近线。

    \item   题目 \cite[page 77, pdf 88, 例5, 例6]{we} 
            通过观察函数图像,辅助确定根的个数,
            有关根的个数请参阅 \ref{number-of-roots-question}。

    \item   题目 \cite[page 107, pdf 118, example 4]{we}
            认识到原函数是偶函数,进而简化求值过程。

\end{itemize}

\subsection{如何联系函数和导数}

TODO
%TODO

\section{Tylor Series and Formula} \label{tylor}

\subsection{公式和形式}

\begin{definition}[泰勒公式的一般形式]
	\begin{equation}
		f(x)=\sum_{i=0}^{n}{\frac{f^{(i)}(x_0)}{i!}(x-x_0)^i}+R_n(x)
	\end{equation}
	其中,$f^{(i)}(x_0)$表示$f(x)$在$x_0$处的$i$阶导数,$R_n(x)$表示余项,有多种形式,例如:
	\begin{align*}
		&R_n(x)=\frac{f^{(n+1)}(\xi)}{(n+1)!}(x-x_0)^{n+1}, \\ &\xi \in [x,x_0]\ or \ \xi \in [x_0,x]
	\end{align*}
\end{definition}

In summary, the Peano remainder focuses on 
the local behavior of the function near the point of expansion 
and gives a rough upper bound on the error for a specific value of $x$. 
\begin{definition}{The Cauchy remainder}
    \[
        R_n=\dfrac{(x-x^*)^n}{n!}(x-x_0)f^{(n+1)}(x^*)
    \]
    where $x^*$ is in $[x_0,x]$ (Hamilton 1952).
\end{definition}
The \textbf{Cauchy remainder provides a more global perspective by 
considering the behavior of derivatives across a broader 
interval between $x_0$ and $x$}. It gives a more refined estimate 
of the error over that interval. This distinction in the scope 
of accuracy makes the Cauchy remainder more useful when you want 
to understand how well the Taylor polynomial approximates the function 
across a larger range of values.

下面是常用的一些Tylor公式
\begin{align}
	\notag e^x&=1+x+\frac{x^2}{2!}+\frac{x^3}{3!} \\ &+\cdots+\frac{x^n}{n!}+o(x^n)\\
	\notag \ln(1+x)&=x-\frac{x^2}{2}+\frac{x^3}{3}\\ &-\cdots+(-1)^{n-1}\frac{x^n}{n}+o(x^n)\\
	\notag \sin x&=x-\frac{x^3}{3!}+\frac{x^5}{5!}\\ &-\cdots+(-1)^{n-1}\frac{x^{2n-1}}{(2n-1)!}+o(x^{2n})\\
	\notag \cos x&=1-\frac{x^2}{2!}+\frac{x^4}{4!}\\ &-\cdots+(-1)^n\frac{x^{2n}}{(2n)!}+o(x^{2n+1})\\
	\notag (1+x)^\alpha&=1+\alpha x+\frac{\alpha(\alpha-1)}{2!}x^2+ \\  &\cdots+\frac{\alpha(\alpha-1)\cdots(\alpha-n+1)}{n!}x^n+o(x^n)
\end{align}

\subsection{做题应用} \label{tylor-app}

其中,\textbf{奇函数}的麦克劳林级数展开中偶数阶导数系数为0。

\section{不等式} \label{inequlity}

\subsection{随听课新增的}

\begin{lemma}
    \[
        \dfrac{x}{1+x} < \ln (1+x) < x
    \]
\end{lemma}



\subsection{基本不等式}

考研数学中常用的不等式有以下几类:

对非负实数$a,b$,有
\begin{equation}
	a+b \geq 2\sqrt {ab}
\end{equation}
等号成立当且仅当$a=b$.

对正实数$a,b$,有
\begin{equation}
	\dfrac{a}{b}+\dfrac{b}{a} \geq 2
\end{equation}
等号成立当且仅当$a=b$.

对任意实数$x,y$,有
\begin{equation}
	(x+y)^2 \leq 2(x^2+y^2)
\end{equation}
等号成立当且仅当$x=y$或$x=-y$.

对任意实数$x,y,z$,有
\begin{equation}
	(x+y+z)^2 \leq 3(x^2+y^2+z^2)
\end{equation}
等号成立当且仅当$x=y=z$.

对任意实数$x_1,x_2,\dots,x_n$和正整数$n$,有
\begin{equation}
	\dfrac{x_1+x_2+\cdots+x_n}{n} \geq \sqrt[n]{x_1x_2\cdots x_n}
\end{equation}
等号成立当且仅当$x_1=x_2=\cdots=x_n$.

对任意实数$a,b,c$和正整数$n$,有
\begin{equation}
	(a^n+b^n+c^n)^3 \geq 27a^nb^nc^n(a+b+c)^n
\end{equation}
等号成立当且仅当$a=b=c$.

对任意实数$a,b,c,d$和正整数$n,m$,有
\begin{equation}
	\begin{array}{c}
		(a^n+b^n+c^n+d^n)(a^m+b^m+c^m+d^m)\\ \geq \\(a^{n+m}+b^{n+m}+c^{n+m}+d^{n+m})^2
	\end{array}
\end{equation}
等号成立当且仅当$a=b=c=d$.

\subsection{三角不等式}
考研数学中常用的三角不等式有以下几种:

\subsubsection{绝对值形式}
对任意实数$a,b$,有
\begin{equation}
	|a-b| \leq |a|+|b|
\end{equation}
等号成立当且仅当$a,b$同号或其中一个为零.

对任意实数$a,b,c$,有
\begin{equation}
	|a+b+c| \leq |a|+|b|+|c|
\end{equation}
等号成立当且仅当$a,b,c$同号或其中两个为零.

对任意实数$a_1,a_2,\dots,a_n$和正整数$n$,有
\begin{equation}
	|a_1+a_2+\cdots+a_n| \leq |a_1|+|a_2|+\cdots+|a_n|
\end{equation}
等号成立当且仅当$a_1,a_2,\dots,a_n$同号或其中$n-1$个为零.

\subsubsection{向量形式}
对$n$维向量 
$\pmb{x}=(x_1,x_2,\dots,x_n),$
$\pmb{y}=(y_1,y_2,\dots,y_n)$,有
\begin{equation}
	|\pmb{x}|-|\pmb{y}|\leqslant |\pmb{x}\pm\pmb{y}|\leqslant |\pmb{x}|+|\pmb{y}|
\end{equation}
等号成立当且仅当$\pmb{x}\parallel\pmb{y}$.

对$n$维向量$\pmb{x}=(x_1,x_2,\dots,x_n),\pmb{y}=(y_1,y_2,\dots,y_n),$
$\pmb{z}=(z_1,z_2,\dots,z_n)$,有
\begin{equation}
	|\pmb{x}+\pmb{y}+\pmb{z}| \leq |\pmb{x}|+|\pmb{y}|+|\pmb{z}|
\end{equation}

\section{巨算符} \label{giant-operator}

\cite[page A36]{stewart}
\[
    \sum \dfrac{f(x)}{g(x)} \neq \dfrac{\sum f(x)}{\sum g(x)}
\]
